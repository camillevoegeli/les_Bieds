% Options for packages loaded elsewhere
\PassOptionsToPackage{unicode}{hyperref}
\PassOptionsToPackage{hyphens}{url}
%
\documentclass[
]{article}
\usepackage{amsmath,amssymb}
\usepackage{iftex}
\ifPDFTeX
  \usepackage[T1]{fontenc}
  \usepackage[utf8]{inputenc}
  \usepackage{textcomp} % provide euro and other symbols
\else % if luatex or xetex
  \usepackage{unicode-math} % this also loads fontspec
  \defaultfontfeatures{Scale=MatchLowercase}
  \defaultfontfeatures[\rmfamily]{Ligatures=TeX,Scale=1}
\fi
\usepackage{lmodern}
\ifPDFTeX\else
  % xetex/luatex font selection
\fi
% Use upquote if available, for straight quotes in verbatim environments
\IfFileExists{upquote.sty}{\usepackage{upquote}}{}
\IfFileExists{microtype.sty}{% use microtype if available
  \usepackage[]{microtype}
  \UseMicrotypeSet[protrusion]{basicmath} % disable protrusion for tt fonts
}{}
\makeatletter
\@ifundefined{KOMAClassName}{% if non-KOMA class
  \IfFileExists{parskip.sty}{%
    \usepackage{parskip}
  }{% else
    \setlength{\parindent}{0pt}
    \setlength{\parskip}{6pt plus 2pt minus 1pt}}
}{% if KOMA class
  \KOMAoptions{parskip=half}}
\makeatother
\usepackage{xcolor}
\usepackage[margin=1in]{geometry}
\usepackage{graphicx}
\makeatletter
\def\maxwidth{\ifdim\Gin@nat@width>\linewidth\linewidth\else\Gin@nat@width\fi}
\def\maxheight{\ifdim\Gin@nat@height>\textheight\textheight\else\Gin@nat@height\fi}
\makeatother
% Scale images if necessary, so that they will not overflow the page
% margins by default, and it is still possible to overwrite the defaults
% using explicit options in \includegraphics[width, height, ...]{}
\setkeys{Gin}{width=\maxwidth,height=\maxheight,keepaspectratio}
% Set default figure placement to htbp
\makeatletter
\def\fps@figure{htbp}
\makeatother
\setlength{\emergencystretch}{3em} % prevent overfull lines
\providecommand{\tightlist}{%
  \setlength{\itemsep}{0pt}\setlength{\parskip}{0pt}}
\setcounter{secnumdepth}{-\maxdimen} % remove section numbering
\ifLuaTeX
  \usepackage{selnolig}  % disable illegal ligatures
\fi
\IfFileExists{bookmark.sty}{\usepackage{bookmark}}{\usepackage{hyperref}}
\IfFileExists{xurl.sty}{\usepackage{xurl}}{} % add URL line breaks if available
\urlstyle{same}
\hypersetup{
  pdftitle={Les Bieds : analyse des gas 2023},
  hidelinks,
  pdfcreator={LaTeX via pandoc}}

\title{Les Bieds : analyse des gas 2023}
\author{}
\date{\vspace{-2.5em}2024-02-19}

\begin{document}
\maketitle

\hypertarget{introduction}{%
\subsection{Introduction}\label{introduction}}

Campagne de mesures : 31.05.2023 - 1.11.2023, 7 mesures

Terrain effectué par Alicia Frésard
(\href{mailto:aliciafresard_ecobio@protonmail.com}{\nolinkurl{aliciafresard\_ecobio@protonmail.com}}),
assistée par Julie Boserup
(\href{mailto:julie.boserup@lineco.ch}{\nolinkurl{julie.boserup@lineco.ch}})

\hypertarget{site}{%
\subsection{Site}\label{site}}

\begin{figure}
\includegraphics[width=0.8\linewidth]{documentation/maps/plots_map_1_2500} \caption{Plan du site}\label{fig:site-1}
\end{figure}
\begin{figure}
\includegraphics[width=0.8\linewidth]{documentation/maps/plan_sites} \caption{Plan du site}\label{fig:site-2}
\end{figure}

Les points de mesures ont été sélectionnés de manière à évaluer les
émissions de gaz dans toutes les environnements présents

\emph{2073:} bas marais à prairie humide, drains désactivés. Fauché Juin
\emph{2517:} prairie. Fauchée Août, Septembre, \emph{2215:} prairie de
fauche. Purinée et fauchée Juin \emph{HM:} haut marais (15= lande,
16=betulaie sur tourbe) \emph{PL:} maraîchage, tourbe nue, plantes
présents en septembre. Sous bâche en novembre

\hypertarget{graphes}{%
\subsection{Graphes}\label{graphes}}

\hypertarget{co2-respiration-toutes-les-placettes}{%
\subsubsection{CO2 respiration, toutes les
placettes:}\label{co2-respiration-toutes-les-placettes}}

\includegraphics{analyse_2023_files/figure-latex/CO2 all-1.pdf}
\includegraphics{analyse_2023_files/figure-latex/boxplot CO2-1.pdf}

à part une diminution des flux en novembre, la tendance à la hausse en
été n'est pas frappante

\begin{verbatim}
## `geom_smooth()` using method = 'loess' and formula = 'y ~ x'
\end{verbatim}

\includegraphics{analyse_2023_files/figure-latex/CO2 site-1.pdf} - La
parcelle 2215, toujours drainée, emet plus de CO2 que les autres. - 2073
et le haut marais montrent une varaition saisonnière plus importante,
probablement dû aux fluctuation de la nappe permises par l'absence de
drains.Le pic de CO2 est dans le plot de land haut marais. Peut être du
à une sécheresse importante de la tourbe, qui se décompose

\begin{itemize}
\item
  2517 ancien lit du ruisseau émettent moins et en décroissance pendant
  la saison
\item
  La parcelle de maraîchage émet mois que les autres
\end{itemize}

\hypertarget{co2-nee-toutes-les-placettes}{%
\subsubsection{CO2 NEE, toutes les
placettes}\label{co2-nee-toutes-les-placettes}}

\begin{figure}
\centering
\includegraphics{analyse_2023_files/figure-latex/NEE all-1.pdf}
\caption{toutes les placettes, CO2, NEE}
\end{figure}

\begin{verbatim}
## `geom_smooth()` using method = 'loess' and formula = 'y ~ x'
\end{verbatim}

\begin{figure}
\centering
\includegraphics{analyse_2023_files/figure-latex/NEE grouped-1.pdf}
\caption{toutes les placettes, CO2, NEE, per site}
\end{figure}

ATTENTION: la lumière jouant un rôle majeur dans l'efficacité de la
photosynthèse, les comparaisons d'un jour à l'autre sont à faire avec
prudence. Il est plus sûr de comparer les différents sites sur un même
journée.

la photosynthèse diminue les emissions de co2 sur toutes les placettes.
le haut marais et 2575 sont en négatif en aout septembre,octobre et
novembre

\includegraphics{analyse_2023_files/figure-latex/PAR-1.pdf}
\includegraphics{analyse_2023_files/figure-latex/methane-1.pdf}

\includegraphics{analyse_2023_files/figure-latex/without may-1.pdf}

Taux très faible voir négatif de ch4, dû à la profondeur de nappe trop
importante. Les emissions ont lieu sur les sites submergés.

le pic negatif correspond au haut marais. Le niveau de nappe est trop
bas pour la production de ch4 mais les methanotrophes sont toujours
actifs

\end{document}
